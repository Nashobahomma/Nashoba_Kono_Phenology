\documentclass[10pt]{standalone} 
\usepackage{booktabs}
%\usepackage{amsmath}
\usepackage{amssymb} %maths
\usepackage{fixltx2e}
%\usepackage[LGRgreek]{mathastext}


\begin{document}

\begin{tabular}{l r r r} 
\toprule
 		& EG		&		& EF  		\\
\midrule
%%%%%%%%%Possibly add G2 estimates for V_E and possibly Wbar as well%%%%%%%%%%%
$V_A$  		& 3.84		& 		& 15.71		\\[5pt]
$V_E$		& 121.9	  	& 		& 75.24		\\[5pt]	
$h^2$		& 0.030		&		& 0.173		\\[5pt]
%$Cov_A$		& 		& 10.58		&		\\[5pt]
%$Cov_E$		&		& 38.18		&		\\[5pt]
$Cor_A$			&		& 1.36		&		\\[5pt]
$Cor_E$			&		& 0.40		&		\\[5pt]

\bottomrule
\end{tabular}

\end{document}

% The values come from Multigenerational Quercus date collected by A.Nashoba, analyzed by T.Kono & A.Nashoba, and run by F.Shaw 
% Source files: FS_Multigen_sibships, mf3mg.p
% Output file: FS_Multigen_Results
% Analyzing environmental and additive genetic variance for combined G1Y13 and G1y14

% EG: Va+VP = full numbers not truncated..3.834964+121.912263=125.747227, h2=3.834964/125.747227=0.030497404
% EF: Va+Vp = 15.706558+75.238494=90.945052,                              h2=15.706558/90.945052=0.172703821

% Now need to calculate Environmental and Additive Genetic Correlations:
%cov / sqrtEG * sqrtEF
%10.58/ sqrt(3.84)*sqrt(15.71) = corA = 1.36
%38.18/ sqrt(121.9)*sqrt(75.24) = corE = 0.40